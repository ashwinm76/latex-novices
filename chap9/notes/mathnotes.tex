% Various math mode notes

\documentclass{scrartcl}

\usepackage{amsmath}
\usepackage{amsfonts}

\newcommand*{\pderiv}[2]{\frac{\partial #1}{\partial #2}}
\newcommand*{\e}{\mathrm{e}}

\begin{document}

\section{Dots and Delimiters}

Various dots:
\begin{align*}
    \text{dotsb:}\ y = a_1 + a_2 + \dotsb + a_n\\
    \text{cdots:}\ y = a_1 + a_2 + \cdots + a_n\\
    \text{ldots:}\ y = a_1 + a_2 + \ldots + a_n\\
    \text{dotsc:}\ y = a_1 + a_2 + \dotsc + a_n\\
    \text{dotsi:}\ y = a_1 + a_2 + \dotsi + a_n\\
    \text{dotso:}\ y = a_1 + a_2 + \dotso + a_n\\
    \text{dotsm:}\ y = a_1 + a_2 + \dotsm + a_n
\end{align*}

Various delimiters:
\begin{align*}
    \text{( ):}   &(\frac{1}{1+x}) \\
    \text{[ ]:}   &[\frac{1}{1+x}] \\
    \text{\{ \}:} &\{\frac{1}{1+x}\} \\
    \text{left lvert, right rvert:} &\left\lvert\frac{1}{1+x}\right\rvert \\
    \text{left lVert, right rVert:} &\left\lVert\frac{1}{1+x}\right\rVert \\
    \text{left langle, right rangle:} &\left\langle\frac{1}{1+x}\right\rangle\\
    \text{left lfloor, right rfloor:} &\left\lfloor\frac{1}{1+x}\right\rfloor\\
    \text{left lceil, right rceil:} &\left\lceil\frac{1}{1+x}\right\rceil\\
    \text{uparrow, downarrow:} &\left\uparrow\frac{1}{1+x}\right\downarrow\\
    \text{Uparrow, Downarrow:} &\left\Uparrow\frac{1}{1+x}\right\Downarrow\\
    \text{updownarrow, updownarrow:} &\left\updownarrow\frac{1}{1+x}\right\updownarrow\\
    \text{/, /:} &\left/\frac{1}{1+x}\right/\\
    \text{\textbackslash backslash, \textbackslash backslash:} &\left\backslash\frac{1}{1+x}\right\backslash\\
\end{align*}

Various delimiter sizess:
\begin{align*}
    \text{( ):}   &(\frac{1}{1+x}) \\
    \text{bigl( bigr):}   &\bigl(\frac{1}{1+x}\bigr) \\
    \text{Bigl( Bigr):}   &\Bigl(\frac{1}{1+x}\Bigr) \\
    \text{biggl( biggr):}   &\biggl(\frac{1}{1+x}\biggr) \\
    \text{Biggl( Biggr):}   &\Biggl(\frac{1}{1+x}\Biggr) \\
\end{align*}

\section{Equations}

Huge equation:
\[
    \pderiv{^2 \mathcal{L}}{{z_i^\rho}^2} = 
    -\pderiv{\rho_i}{z_i^\rho}
    \left(
        \pderiv{v_i}{\rho_i}\frac{\e^{v_i}}{1-\e^{v_i}}
        +v_i\frac
        {\e^{v_i}\pderiv{v_i}{\rho_i}\left(1-\e^{v_i}\right) + \e^{2v_i}\pderiv{v_i}{\rho_i}}
        {(1-\e^{v_i})^2}
    \right)
\]

\section{Arrays}

Basic arrays:

Using \emph{array}:
\[
\begin{array}{rrr}
    1 & 2 & 3\\
    4 & 5 & 6\\
    7 & 8 & 9
\end{array}
\]

The same but with \emph{pmatrix} from \emph{amsmath}:
\[
\begin{pmatrix}
    1 & 2 & 3\\
    4 & 5 & 6\\
    7 & 8 & 9
\end{pmatrix}
\]

The same but with \emph{bmatrix} from \emph{amsmath}:
\[
\begin{bmatrix}
    1 & 2 & 3\\
    4 & 5 & 6\\
    7 & 8 & 9
\end{bmatrix}
\]

The same but with \emph{Bmatrix} from \emph{amsmath}:
\[
\begin{Bmatrix}
    1 & 2 & 3\\
    4 & 5 & 6\\
    7 & 8 & 9
\end{Bmatrix}
\]

The same but with \emph{vmatrix} from \emph{amsmath}:
\[
\begin{vmatrix}
    1 & 2 & 3\\
    4 & 5 & 6\\
    7 & 8 & 9
\end{vmatrix}
\]

The same but with \emph{Vmatrix} from \emph{amsmath}:
\[
\begin{Vmatrix}
    1 & 2 & 3\\
    4 & 5 & 6\\
    7 & 8 & 9
\end{Vmatrix}
\]

Arrays with delimiters:
\[
\left(
    \begin{array}{rrr}
        1 & 2 & 3\\
        4 & 5 & 6\\
        7 & 8 & 9
    \end{array}
\right)
\]

Arrays with vertical rules:
\[
\begin{array}{rr|r}
    1 & 2 & 3\\
    4 & 5 & 6\\
    7 & 8 & 9
\end{array}
\]

Arrays with invisible delimiters (the right delimiter is invisible):
\[
    f(x) = 
    \left\{
    \begin{array}{rl}
        -1 & x < 0\\
         0 & x = 0\\
        +1 & x > 0
    \end{array}    
    \right.
\]

Writing the above using the \emph{amsmath} cases environment:
\[
    f(x) = 
    \begin{cases}
        -1 & x < 0\\
         0 & x = 0\\
        +1 & x > 0
    \end{cases}
\]

Inline matrices are written using the \emph{smallmatrix} environment
\begin{math}
    \left(
        \begin{smallmatrix}
            1 & 2 & 3\\
            4 & 5 & 6
        \end{smallmatrix}
    \right)
\end{math}
within the line of text. Hoe many rows are realistic?
\begin{math}
    \left(
        \begin{smallmatrix}
            1 & 2 & 3\\
            4 & 5 & 6\\
            7 & 8 & 9
        \end{smallmatrix}
    \right)
\end{math}

\section{Vectors}

Here is a vector: $\vec{v}$.

Redefining the \emph{vec} command to typeset vectors in bold (without the arrow
above it), as is the convention in some texts:

\let\origvec\vec

\renewcommand{\vec}[1]{\mathbf{#1}}

\[
    \vec{x}\cdot\vec{\xi} = z
\]

The \emph{boldsymbol} command from the \emph{amsfonts} package can be used to ensure
that the greek letters are bold as well. Note however that a side effect of this
command is that the result is both bold and italic:

\renewcommand{\vec}[1]{\boldsymbol{#1}}

\[
    \vec{x}\cdot\vec{\xi} = z
\]

\let\vec\origvec

Default vector arrow (which can be too small) vs the \emph{overrightarrow} command:
\[
    \vec{OP}\ \text{vs}\ \overrightarrow{OP}
\]

\end{document}