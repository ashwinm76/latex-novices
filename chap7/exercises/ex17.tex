% Exercise 17 (Creating Tables)

\documentclass[12pt]{scrbook}

% datetime options:
% short: abbreviated format
% nodayofweek: don't display the day of the week
% level: put the "th" at the same level as the text.
% 12hr or 24hr: select 12 or 24 hour format
%\usepackage[short,nodayofweek,level,12hr]{datetime}
%\usepackage[short,nodayofweek,level,24hr]{datetime}
%\usepackage[short,nodayofweek,12hr]{datetime}
%\usepackage[nodayofweek,12hr]{datetime}
\usepackage[short,12hr]{datetime}

\title{A Simple Document}
\author{Ashwin Menon}

\begin{document}

\maketitle
\frontmatter
\listoftables

\mainmatter
This is a simple \LaTeX\ document. Here is the first paragraph.

Here is the second paragraph\footnote{with a footnote}. As you
can see it's a rather short paragraph, but not as short as the
previous one. This document was created on: \today\ at \currenttime.

Table \ref{tab:travel_expenses} shows travel expenses.
\begin{table}[htbp]
    \caption{Travel Expenses}
    \label{tab:travel_expenses}
    \centering
    \begin{tabular}{lrr}
                        & \multicolumn{2}{c}{\bfseries Expenditure (\pounds)} \\
                        & \multicolumn{1}{c}{Year1} & \multicolumn{1}{c}{Year2} \\
    \bfseries Travel    & 100,000 & 110,000 \\
    \bfseries Equipment & 50,000  & 60,000
    \end{tabular}
\end{table}

Table \ref{tab:sample_table} is a sample table.
\begin{table}[htbp]
    \caption{A Sample Table}
    \label{tab:sample_table}
    \centering
    \begin{tabular}{lr}
        Item  & Cost \\
        Video & 8.99 \\
        CD    & 9.99 \\
        DVD   & 15.00 \\
    \end{tabular}
\end{table}

\end{document}