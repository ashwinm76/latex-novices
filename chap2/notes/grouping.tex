\documentclass{book}

\begin{document}

% When commands appear in a group {}, they apply to the whole group. So in the
% first line below, the bold applies to the whole group, resulting in a bold
% phrase followed by a bold-italic phrase. In the second line below, the bold
% and italic commands appear in separate groups and are restricted to their
% respective groups.
{\bfseries Some bold text \itshape and some bold-italic text.}

{\bfseries Some bold text} {\itshape and some italic text.}

% The first non-space object after a command is an argument. So in the first
% command below only the S is bold, whereas in the second command below the
% entire phrase is bold because the whole group is the argument.
\textbf Some bold text.
\textbf{Some bold text.}

% Empty arguments can be specified by {}. This is useful: spaces immediately
% following a command are ignored, like this:
\TeX nician

% To force a space after a command, use {}, like this:
\TeX{} nician

% {} is also used if you want to give no argument where one is required. E.g.
% \chapter requires the chapter title as an argument, so \chapter{} results
% in a chapter heading with no title.
\chapter{Needs a title} % Title is mandatory
\chapter{} % Title omitted by being empty

% Optional arguments are in []. E.g. the \\ command has an optional argument
% to specify the gap:
Line 1 \\Line 2

Line 3 followed by a 1cm gap \\[1cm]Line 4

% Multiple optional arguments are each in []. E.g. right-justified text in a
% 5cm wide box:
\framebox[5cm][r]{Text in a box.}

\end{document}