% Exercise 26 (Using Counters)

\documentclass[12pt]{scrreprt}

% datetime options:
% short: abbreviated format
% nodayofweek: don't display the day of the week
% level: put the "th" at the same level as the text.
% 12hr or 24hr: select 12 or 24 hour format
%\usepackage[short,nodayofweek,level,12hr]{datetime}
%\usepackage[short,nodayofweek,level,24hr]{datetime}
%\usepackage[short,nodayofweek,12hr]{datetime}
%\usepackage[nodayofweek,12hr]{datetime}
\usepackage[short,12hr]{datetime}

\newcounter{exercise}
\renewcommand{\theexercise}{\thechapter.\arabic{exercise}}

\newenvironment{exercise}[1]
{
    \par\vspace{\baselineskip}\noindent
    \refstepcounter{exercise}
    \textbf{Exercise \theexercise\ - #1}\begin{itshape}
    \par\vspace{\baselineskip}\noindent\ignorespaces
}
{
    \end{itshape}\ignorespacesafterend
}

\title{A Simple Document}
\author{Ashwin Menon}

\begin{document}

\maketitle

\begin{abstract}
A brief document to illustrate how to use \LaTeX.    
\end{abstract}

\chapter{Introduction}
\setcounter{exercise}{0}

\section{The First Section}
This is a simple \LaTeX\ document. 
Here is the first paragraph.

\section{The Next Section}
Here is the second paragraph\footnote{with a footnote}. As you
can see it's a rather short paragraph, but not as short as the
previous one. This document was created on: \today\ at \currenttime.

Running (see Exercise~\ref{ex:run}) is great exercise.

\begin{exercise}{Run}
    \label{ex:run}
This is the first exercise. Go outside and run for 30 minutes.
\end{exercise}

\chapter{Another Chapter}

\setcounter{exercise}{0}

Here's another very interesting chapter.
We're going to put a picture here later.

If you want strong shoulders, do push-ups (see Exercise~\ref{ex:push_ups}).

\begin{exercise}{Push ups}
    \label{ex:push_ups}
This is the second exercise. Perform 10 push-ups.
\end{exercise}

\chapter*{Acknowledgements}

I would like to acknowledge all those
very helpful people who have assisted me in my work,

\appendix

\chapter{Tables}
We will turn this tabular environment into a table later.

\begin{tabular}{lrr}
                     & \multicolumn{2}{c}{\bfseries Expenditure (\pounds)} \\
                     & \multicolumn{1}{c}{Year1} & \multicolumn{1}{c}{Year2} \\
 \bfseries Travel    & 100,000 & 110,000 \\
 \bfseries Equipment & 50,000  & 60,000
\end{tabular}

\end{document}
