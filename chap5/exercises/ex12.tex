% Exercise 12 (Cross-Referencing)

\documentclass[12pt]{scrreprt}
%\documentclass[12pt, toc=flat]{scrreprt}

% datetime options:
% short: abbreviated format
% nodayofweek: don't display the day of the week
% level: put the "th" at the same level as the text.
% 12hr or 24hr: select 12 or 24 hour format
%\usepackage[short,nodayofweek,level,12hr]{datetime}
%\usepackage[short,nodayofweek,level,24hr]{datetime}
%\usepackage[short,nodayofweek,12hr]{datetime}
%\usepackage[nodayofweek,12hr]{datetime}
\usepackage[short,12hr]{datetime}

\title{A Simple Document}
\author{Ashwin Menon}

\begin{document}

\maketitle

\tableofcontents

\begin{abstract}
A brief document to illustrate how to use \LaTeX.    
\end{abstract}

\chapter{Introduction}
\label{ch:intro}

\section{The First Section}
This is a simple \LaTeX\ document. Here is the first paragraph.
The next chapter is Chapter~\ref{ch:another}
and is on page~\pageref{ch:another}.
The next section is Section~\ref{sec:next}.

\section{The Next Section}
\label{sec:next}

Here is the second paragraph\footnote{with a footnote}. As you
can see it's a rather short paragraph, but not as short as the
previous one. This document was created on: \today\ at \currenttime.

\chapter{Another Chapter}
\label{ch:another}

Here's another very interesting chapter.
We're going to put a picture here later.
See Chapter~\ref{ch:intro} for an introduction.

\chapter*{Acknowledgements}

I would like to acknowledge all those
very helpful people who have assisted me in my work,

\appendix

\chapter{Tables}
We will turn this tabular environment into a table later.

\begin{tabular}{lrr}
                     & \multicolumn{2}{c}{\bfseries Expenditure (\pounds)} \\
                     & \multicolumn{1}{c}{Year1} & \multicolumn{1}{c}{Year2} \\
 \bfseries Travel    & 100,000 & 110,000 \\
 \bfseries Equipment & 50,000  & 60,000
\end{tabular}

\end{document}
